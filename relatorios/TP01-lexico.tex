\documentclass[11pt]{article}

\usepackage[brazil]{babel}
\usepackage[utf8]{inputenc}
\usepackage{graphicx}
\usepackage{multirow}
\usepackage{subfigure}
\usepackage{a4wide}
\usepackage{fancyhdr}
\usepackage{algorithm}
\usepackage{algorithmic}
\usepackage{tikz}
\usepackage{empheq}
\usetikzlibrary{trees}
\usepackage{multirow}
\usepackage{subfigure}
\usepackage{amssymb,amsmath}
\usepackage{amsthm,amsfonts}
\usepackage{float}
\graphicspath{ {images/} }
\usepackage{listings}
\usepackage{color}

\pagestyle{fancy}
\renewcommand{\headrulewidth}{0.1pt}
\renewcommand{\footrulewidth}{0.1pt}



\definecolor{mygreen}{rgb}{0,0.6,0}
\definecolor{mygray}{rgb}{0.5,0.5,0.5}
\definecolor{mymauve}{rgb}{0.58,0,0.82}
\lstset{ %
  backgroundcolor=\color{white},   % choose the background color; you must add \usepackage{color} or \usepackage{xcolor}
  basicstyle=\footnotesize,        % the size of the fonts that are used for the code
  breakatwhitespace=false,         % sets if automatic breaks should only happen at whitespace
  breaklines=true,                 % sets automatic line breaking
  columns=flexible,
  captionpos=b,                    % sets the caption-position to bottom
  commentstyle=\color{mygreen},    % comment style
  %deletekeywords={...},           % if you want to delete keywords from the given language
  %escapeinside={\%*}{*)},         % if you want to add LaTeX within your code
  extendedchars=true,              % lets you use non-ASCII characters; for 8-bits encodings only, does not work with UTF-8
  frame=leftline,	               % adds a frame around the code
  keepspaces=true,                 % keeps spaces in text, useful for keeping indentation of code (possibly needs columns=flexible)
  keywordstyle=\color{blue},       % keyword style
  language=Make,                   % the language of the code
  %otherkeywords={*,...},          % if you want to add more keywords to the set
  numbers=left,                    % where to put the line-numbers; possible values are (none, left, right)
  numbersep=10pt,                  % how far the line-numbers are from the code
  numberstyle=\tiny\color{mygray}, % the style that is used for the line-numbers
  %rulecolor=\color{black},        % if not set, the frame-color may be changed on line-breaks within not-black text (e.g. comments (green here))
  showspaces=false,                % show spaces everywhere adding particular underscores; it overrides 'showstringspaces'
  showstringspaces=false,          % underline spaces within strings only
  showtabs=false,                  % show tabs within strings adding particular underscores
  stepnumber=1,                    % the step between two line-numbers. If it's 1, each line will be numbered
  stringstyle=\color{mymauve},     % string literal style
  tabsize=4,	                   % sets default tabsize to 2 spaces
  %title=\lstname                  % show the filename of files included with \lstinputlisting; also try caption instead of title
  xleftmargin=1cm
}




\lhead{Analisador Léxico e Tabela de Símbolos}
\rhead{\thepage} 
\lfoot{ Felipe Duarte, Gustavo Borba, Juan Lopes }
\rfoot{ \today }
\cfoot{}


\begin{document}
\thispagestyle{empty}

\begin{center}
\begin{minipage}[l]{10cm}{
\center
Compiladores \\
2016/01 \\
}\end{minipage}
 \vfill
 \begin{minipage}[l]{11cm}{
   \begin{center}
   \Large{Trabalho Prático 01: \\ Analisador Léxico e Tabela de Símbolos}
   \end{center}
}\end{minipage}
\end{center}
 \vspace*{8cm}
 \begin{center}
 \begin{minipage}[l]{10cm}{
 \center Felipe Duarte, Gustavo  Borba, Juan Lopes\\
 Belo Horizonte, \today \\
 }
 \end{minipage}
 \end{center}

\newpage
\thispagestyle{empty}
\tableofcontents

\newpage
\clearpage
\setcounter{page}{1}

\section{Introdução}
	O programa pas-c é um projeto desenvolvido na aula de Compiladores do Curso de Engenharia da Computação do CEFET-MG. Pas-c é a linguagem homônima, definida para esse compilador, cuja semântica dos comandos e expressões é a tradicional de linguagens como Pascal e C (daí seu nome).
	
	\subsection{A Linguagem Pas-c}

		Definimos a gramática da linguagem:
		
		\begin{lstlisting}
program ::= [ var decl-list] begin stmt-list end
decl-list ::= decl ";" { decl ";"}
decl ::= ident-list is type
ident-list ::= identifier {"," identifier}
type ::= int | string
stmt-list ::= stmt ";" { stmt ";"}
stmt ::= assign-stmt | if-stmt | do-stmt | read-stmt | write-stmt
assign-stmt ::= identifier ":=" simple_expr
if-stmt ::= if condition then stmt-list end 
			| if condition then stmt-list else stmt-list end
condition ::= expression
do-stmt ::= do stmt-list stmt-suffix
stmt-suffix ::= while condition
read-stmt ::= in "(" identifier ")"
write-stmt ::= out "(" writable ")"
writable ::= simple-expr
expression ::= simple-expr | simple-expr relop simple-expr
simple-expr ::= term | simple-expr addop term
term ::= factor-a | term mulop factor-a
fator-a ::= factor | not factor | "-" factor
factor ::= identifier | constant | "(" expression ")"
relop ::= "=" | ">" | ">=" | "<" | "<=" | "<>"
addop ::= "+" | "-" | or
mulop ::= "*" | "/" | and
constant ::= integer_const | literal
integer_const ::= nozero {digit} | "0"
literal ::= " {" {caractere} "}"
identifier ::= (letter) {letter | digit } 
				| "_" (letter | digit ) {letter | digit }
letter ::= [A-Za-z]
digit ::= [0-9]
nozero ::= [1-9]
caractere ::= um dos 256 caracteres do conjunto ASCII, 
				exceto "{", "}" e quebra de linha
		\end{lstlisting}

\newpage
\section{Uso do compilaodor}

	
	\subsection{Compilando o Compilador Pas-c}

		Para compilar o seu código (neste momento, apenas como analizador léxico), basta ter um compilador G++ versão 4.8 ou superior com cmake instalado em sua máquina.
		
		Na pasta do projeto execute o comando:
\begin{lstlisting}[language=C++]
make
\end{lstlisting} 
		
		Esse comando deverá gerar a seguinte seqûencia de comandos:
\begin{lstlisting}[language=C++]
g++ -Iinclude -c -Wall -g -DRUN_TESTS src/Scanner.cpp -o src/Scanner.o
g++ -Iinclude -c -Wall -g -DRUN_TESTS src/TestCase.cpp -o src/TestCase.o
g++ -Iinclude -c -Wall -g -DRUN_TESTS src/Token.cpp -o src/Token.o
g++ -Iinclude -c -Wall -g -DRUN_TESTS src/main.cpp -o src/main.o
g++  src/Scanner.o src/TestCase.o src/Token.o src/main.o -o pasc
\end{lstlisting}
		
		Pronto! Agora o executável \textbf{pasc} está pronto para compilar seus programas em Pas-c!
	
	
	\subsection{Compilando um programa em Pas-c}
	
		Para executar o compilador, basta executar o comando:
		
		\begin{verbatim}
			pasc caminho_para_seu_codigo.pasc
		\end{verbatim}
	

\newpage
\section{A Implementação do Compilador}
	
	\subsection{A abordagem utilizada na implementação}
		
		O compilador \textbf{pas-c} utiliza um analisador léxico recursivo, \textbf{e quem vai saber explicar essas parada toda é o japão}, eu nem estudei pra segunda prova ainda. 
		
	\subsection{A Estrutura do Projeto}
		
		O projeto segue a estrutura básica de todo projeto em C++, contendo uma pasta \textbf{include} e, dentro desta, separados por módulos, os headers (.h) com as declarações das classes. A pasta \textbf{src} contém a implementação das classes, bem como a função principal \textbf{main}:
		
		\begin{verbatim}
			include/ -- Pasta para os headers
			include/frontend -- pasta contendo as definições para os analisadores.
			include/backend -- ainda por fazer.
			include/test -- Pasta com os headers referentes aos testes unitários
			src/ -- Implementação das classes
			tests/ -- Pasta com a implementação dos testes unitários 
		\end{verbatim}
		
		
		
	\subsection{Principais Classes da Aplicação}
		
		
		\begin{itemize}
			
			\item \textbf{Token: }A classe Token descreve a unidade lógica mais básica do compilador. Ela apresenta apenas um valor, em formato de cadeia de caracteres, e seu \textbf{Tipo}, a ser elucidado no próximo item. 
			
			\item \textbf{TokenType: } TokenType trata-se apenas de um enum, que define, através de um bitmap, a lista de tokens reconhecidos pelo compilador, bem como um tipo UNKNOWN, para todo e qualquer token que não corresponder a nenhuma definição da linguagem.
			
			\item \textbf{Scanner: } O analisador léxico está concentrado basicamente na classe Scanner. Essa classe contém os métodos \textbf{getNumerical}, \textbf{getLiteral} e \textbf{getOperator}, que são responsáveis por captar os tokens, bem como o método \textbf{getString}. Este último é necessário por que uma vez que se abre um caracter delimitador de string, espacos em brancos passam a ser caracteres significativos. Também estão definidos nessa classe os contadores de linhas e colunas do código-fonte.
		
		\end{itemize}
		

\newpage
\section{Resultados dos testes especificados}


	\subsection{Teste 1}
	
		Código fonte testado:
		\lstinputlisting[language=Pascal]{../tests/test1.psc}
			
		Saída encontrada:
		\lstinputlisting{../tests/output/test1.out}
	
	
	\subsection{Teste 2}
		
		Código fonte testado:
		\lstinputlisting[language=Pascal]{../tests/test2.psc}
			
		Saída encontrada:
		\lstinputlisting{../tests/output/test2.out}


	\subsection{Teste 3}
	
		Código fonte testado:
		\lstinputlisting[language=Pascal]{../tests/test3.psc}
			
		Saída encontrada:
		\lstinputlisting{../tests/output/test3.out}


	\subsection{Teste 4}
	
		Código fonte testado:
		\lstinputlisting[language=Pascal]{../tests/test4.psc}
			
		Saída encontrada:
		\lstinputlisting{../tests/output/test4.out}
		
	
	\subsection{Teste 5}
	
		Código fonte testado:
		\lstinputlisting[language=Pascal]{../tests/test5.psc}
			
		Saída encontrada:
		\lstinputlisting{../tests/output/test5.out}


	\subsection{Teste 6}
	
		Código fonte testado:
		\lstinputlisting[language=Pascal]{../tests/test6.psc}
			
		Saída encontrada:
		\lstinputlisting{../tests/output/test6.out}


	\subsection{Teste 7}
	
		Código fonte testado:
		\lstinputlisting[language=Pascal]{../tests/test7.psc}
			
		Saída encontrada:
		\lstinputlisting{../tests/output/test7.out}


	\subsection{Teste 8}
	
		Código fonte testado:
		\lstinputlisting[language=Pascal]{../tests/test8.psc}
		
		Saída encontrada:
		\lstinputlisting{../tests/output/test8.out}



\end{document} 